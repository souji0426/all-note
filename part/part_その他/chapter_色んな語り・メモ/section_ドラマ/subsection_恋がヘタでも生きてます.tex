\documentclass[C:/souji/all-note/note]{subfiles}

\begin{document}
Huluで1月1日の早朝から一気に全話見てしまった。
原作は見たことはない。
展開としてはありがちで、
とくに気になるキャラやストーリーではなかった。
それゆえに作業しながらでも見れたのかもしれない。
Wikipediaは\cite{Wiki013}として登録しておいた。

しかしいくつかこれまでの経験上感情移入してしまう場面はあった。
1つ目は律儀にセフレとして立ち回ろうとする榎田 千尋だろうか。
わざわざ正しい?セフレ像なんて調べて実行してしまうところが、
いくら相手が提案した関係性とはいえ、
素直というか愚かというか。
しかしそれくらいしてまで関係を絶ちたくないという気持ちを持てることは個人的には羨ましかったりする。
というか千尋の元彼氏はあまり報われない結果に終わったしまったのが残念(もちろん同情されるような行為はそんなにしていないのだけど)。

もう1つは第7話の千尋と橋本がセフレになったシーンだろうか。
行為が終わった後に、
以前に千尋が服を畳んでくれたことを思い出す場面は、
同様の経験はないもののなぜか懐かしいような、
以前感じたような気持ちになってしまった。
女性からされたことは意外とその瞬間にはありがたみも湧かなくて、
そうでない場面のふとしたときに思い出すもの。
そしてなんでしてくれなくなったのだろうと気になってしまう。
ここで橋本はセフレという関係に今まで通りにのめり込めなくなってしまったのだろうと想像する。
もちろん彼はあまりにも気を遣われるのが嫌いらしいので、
何も服を畳むことを求めているわけではないとは思う。
でも気になってしまったんだろうなと。

結局主人公は恋愛主体のドラマの王道といった感じ。
それゆえに徐々に違う深みにハマっていくもう1人の主人公が際立つのかもしれない。
だからこそ自分の印象にもより深く残ったのかも。

見出してから嬉しくなったポイントは、
このドラマに限らないが知っている俳優が増えたことによって、
過去作のキャラと比較して楽しめるようになったことだろうか。
このドラマでも、
Huluを契約するきっかけになった『来世ではちゃんとします』(その語りは\ref{katari:来世ではちゃんとします})の主役だった、
内田理央(Wikipediaは\cite{Wiki014})と、
小関裕太(Wikipediaは\cite{Wiki015})が出演している。
内田は性に奔放という点では『来世』でも同じだが、
性に対しての主体性が全く異なる。
『来世』では振り回されることが多いのに対し、
このドラマでは悪女というか完全に振り回す側になっている。
ただそれも千尋を意識し始めた橋本には通じなかったり、
最終回周辺では千尋の良きライバルのような立ち位置になったりと、
見どころはたくさんあった。
小関も『来世』とでは女性への接し方が真逆。
『来世』では女性のことを軽んじているが、
このドラマでは恋した女性に大してかなり献身的というか健気であった。
二人のキャラがそれぞれ真逆になっているのは面白かった。
ちなみに放送順としては『来世』の方が後。

あとは相棒シリーズで好きだった寺脇康文(Wikipediaは\cite{Wiki016})を久々に見れたのは嬉しかった。
相変わらず亀山君っぽいのは変わらないが、
このドラマの役柄では、
色んな経験をしつつも飄々としているようなキャラであり、
友だちとして一緒にいたら楽しそうだと思う。
最終回付近で主人公に再度考えさせる場面は良かった。
恋愛ものとしては王道展開のような終盤だったけど、
やはり彼は王道展開がよく似合う。

あと俳優ではないが、
占い師役で登場した、
漫才コンビの相席スタート(Wikipediaは\cite{Wiki018})の山﨑 ケイは、
登場がいきなりだったのでビックリした(このときだけのちょい役なので仕方ないけれど)。
あとこのコンビも漫才含めて最近見てなったので登場が嬉しかった。
また漫才も見てみよう。

そして主題歌も耳に残るもので、
各話の終盤を盛り上げてくれる。
音楽には疎いので知らなかったのだが、
秦 基博(はた もとひろ)というシンガーソングライター(Wikipediaは\cite{Wiki017})の『Girl』という曲。
とくにアルバムを聞いてみたりはしないが、
また違う場面で彼の曲に出会いたい。

以上、
ほとんど流し見といった視聴だったが、
別に不快になるわけでもなくそれなりに楽しませてくれた。

\end{document}
