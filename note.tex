\documentclass[a4j,dvipdfmx,10pt]{jbook}

%プリアンブル情報開始
\usepackage{C:/souji/all-note/preamble/souji_package}
%必要なパッケージをまとめたもの
\usepackage{C:/souji/all-note/preamble/souji_macro}
%作ったマクロをまとめたもの
\usepackage{C:/souji/all-note/preamble/souji_thm_style}
%自前の定理環境設定をまとめたもの
%\usepackage{C:/souji/all-note/preamble/souji_xr}
%いらないかも

%目次作るために必要
\makeindex

%もし画像を挿入するなら設定
%\graphicspath{{C:/math/note-logic-intro/pic/}}

%見栄え関係
%余白の設定。各cm分余白が削られる
\usepackage{geometry}
\geometry{top=2cm, bottom=2cm, left=1cm, right=1cm, includefoot}
%\usepackage[margin=15mm]{geometry}

\setlength{\parindent}{0pt}
%段落最初のインデント量を0にしている

%ヘッダーに入れる情報
\usepackage{fancyhdr}
\pagestyle{fancy}
\lhead{soujiノート}
\rhead{\bf\thepage}


%ノートの名前や作者、出力日時の設定
\title{soujiノート}
\author{souji}
\date{}

%プリアンブル終了

\begin{document}

\frontmatter

\maketitle
%タイトルを出力

\input{./ノートの説明文.tex}
%このノートの説明文

\setcounter{tocdepth}{2}
%目次にsubsectionまで表示させる設定
\tableofcontents
%目次の出力

\mainmatter

\part{研究}\label{part:研究}

\part{学習}\label{part:学習}

\part{その他}\label{part:その他}

%\setcounter{chapter}{-1}
%0章から開始
%\chapter{導入}\label{chapter:導入}



%ここからおまけ
%\appendix


\backmatter
%include{atogaki}

%\chapter*{記号の索引}\label{chapter:記号の索引}

\renewcommand{\bibname}{参考文献}
%設定しないと「関連図書」になる

\bibliographystyle{jplain}
\bibliography{C:/math/my_library/bibliography/souji.bib}

\printindex
%索引を出力

\end{document}
